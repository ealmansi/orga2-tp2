% ** Cuidado: El indice se actualiza al compilar 2 veces. **

% -- Configuración --
\documentclass[10pt, a4paper]{article}
\usepackage[paper=a4paper, left=1.5cm, right=1.5cm, bottom=1.5cm, top=1.5cm]{geometry}
\usepackage[utf8]{inputenc}
\usepackage[spanish]{babel}
\usepackage{float}
\usepackage{mdframed}
% \usepackage[colorlinks=true, linkcolor=blue]{hyperref}
\usepackage{graphicx}
\usepackage{titling}
\usepackage{verbatim}
\usepackage{minted}
\usepackage{listings}
\usepackage{color}
\usepackage{tikz}
\usepackage{dsfont}
\usepackage{amsmath}
\usetikzlibrary{chains,fit,shapes}
\newcommand{\ig}[1]{\includegraphics[width=\textwidth]{#1}}
\newcommand{\RNum}[1]{\uppercase\expandafter{\romannumeral #1\relax}}
\usepackage{parskip}
\usepackage{handy}
\usepackage{listings}
\usepackage{mdframed}

 
% -- Documento --
\begin{document}

\setlength{\parindent}{25pt}

	% -- Carátula --
	
	\thispagestyle{empty}
	\title{%
	\huge{Organización del Computador \RNum{2}}\\
	\vspace{4mm}
	\large{Trabajo Práctico 2}
	}
	\date{\vspace{5mm}$2^{\mathrm{do}}$ cuatrimestre de 2013}

	\author{
		\\
		{\rm Emilio Almansi }\\
		\small{ealmansi@gmail.com}
		\and
		\\
		{\rm Miguel Duarte}\\
		\small{miguelfeliped@gmail.com}
	} % end author
	\maketitle

	
	\vspace{10mm}
	% -- Resumen --
	\begin{abstract}
		\centering{%
			En este trabajo se estudia el paradigma de paralelismo a nivel de datos, o el modelo SIMD (\textbf{S}ingle \textbf{I}nstruction - \textbf{M}ultiple \textbf{D}ata), analizando las diferencias en rendimiento y dificultad de desarrollo encontradas al implementar distintos filtros de imágenes y video en lenguaje C y en lenguaje ensamblador mediante el soporte para dicho paradigma provisto por la arquitectura Intel. Adicionalmente, se analizan diferentes optimizaciones disponibles en compiladores típicos, así como estrategias algorítmicas de optimización como desenrollado de ciclos y entubado de código.
		}
	\end{abstract}
	\vspace{10mm}

	% -- Indice --

	\tableofcontents

	\newpage
	% -- Contenido --
	% Se edita cada seccion por separado, no hay que cambiar nada acá.
	\section{Introducción}
		% Explicar en que consitió el trabajo (brevemente).
	% - Explicar levemente los criterios usados para la paralelización
	% - Por qué es buena idea usar este tipo de procesamiento en filtros de imágenes. (todos los pixeles reciben exactamente el mismo procesamiento y así en una sola ráfaga se peude levantar y procesar muchos).
	% - Explicar la hipótesis (debería ser n veces mas rápido porque zarasa).
	% - Se intentó estudiar si vale la pena o no el esfuerzo adicional de desarrollo y debuggeo.


	A lo largo de este documento se realizará un análisis consiso sobre el
procesamiento de datos SIMD mediante las intrucciones SSE de la arquitectura
AMD-64. El mismo se hará mediante la comparación de diversas implementaciones
de 3 filtros (2 de video y uno de imágen).

	El procesamiento SIMD consiste en realizar una misma operación sobre varios
datos de manera simultánea. Es decir que lo que se logra es paralelismo a nivel
de datos. Este tipo de procesamiento resulta especialmente conveniente para una
gran variedad de procesos multimedia en los cuales un mismo proceso se aplica
a muchos datos y los datos se pueden trabajar en bloque, como los filtros
implementados en este trabajo. Existen también procesos que muestran las
limitaciones de este esquema de procesamiento como la lupa con cosenos y
la rotación por ejemplo. En general el motivo de estas limitaciones
consiste en que en alguna parte del algorítmo los datos ya no se pueden
tratar mas como un bloque. Por ejemplo en el algoritmo de rotación
uno trae datos consecutivos, los procesa y a la hora de guardarlos
estos no oscupan posiciones consecutivas, por lo tanto a la hora
de grabar los datos no es posible hacerlo utilizando un proceso SIMD.

\begin{figure}[h]
\begin{center}
  \includegraphics[scale=0.4]{secciones/introduccion/imagenes/SISD.png}
    \includegraphics[scale=0.4]{secciones/introduccion/imagenes/SIMD.png}
\end{center}
\caption{Esquemas de procesamiento SISD y SIMD}
\label{fig:SISD-SIMD}
\end{figure}


	Sin embargo para realizar este análisis es imperioso meterse un poco mas adentro.
El comportamiento esperado a priori para estos procesos es una relación inversa entre
el nivel de paralelismo y el tiempo consumido. Es decir, al procesar 4 datos a la vez
uno esperaría obtener que el proceso tarde 4 veces menos tiempo, procesando 8 datos
a la vez 8 veces menos tiempo, etc. Sin embargo esto no siempre ocurre. Durante
este trabajo constantemente se va a intentar explicar que desviaciones se 
produjeron con respecto a este supuesto. Para eso vamos a analizar la arquitectura
intel-64, la velocidad de acceso a memoria, el modo en que se usa la caché,
el medio en el cuál se ejecutan los programas y los algoritmos utilizados entre
otras cosas.

	A lo largo del trabajo se va a ir mostrando como el uso de código de ensamblador
optimizado para el uso de la tecnología SSE produce programas sumamente eficientes. Sin embargo
producir ese código es sumamente trabajoso, mucho mas que usar lenguajes de mas alto nivel como
c o c++. Por ese motivo se intentará hacer otro análisis (tal vez algo menos científico
pero intentando justificar de la manera mas objetiva posible) de cuando vale la pena y cuando no.


	\newpage

	\section{Consideraciones generales}
		% comparamos implemtaciones del mismo filtro (versión mas intuitiva posible de c vs asm con SIMD). Esto permite analizar mejor la optimizaciones del compilador. Sobre todo las de icc que tiene introducción automática de SIMD
% En todos los casos se realizó el gráfico C vs ASM (cantidad de clocks y cantidad de líneas de código).
% Como medimos. (menor vs promedio)
% Como se compiló. Compilamos con gcc, con icc, y con diferentes flags.
	% - (aclarar cuales y por qué)
	% - Explicar por qué se puso ICC y no otro(introduce SIMD automáticamente y se supone que es el compilador mas optimo para la arquitectura AMD-64 con procesadores intel).
% cómo se ejcutó. (Mínimas interrupciones, init 2 y kill -9 -1).
% Análisis del código. (object dump).
% - Los tres filtros se programaron pensando en minizar los accesos a memoria incluso en situaciones donde producía cálculo extra.

A la hora de implementar los filtros se tuvieron algunas consideraciones generales para mantener la coherencia dentro del trabajo y poder extraer resultados comparables entre los distintos filtros, como se detallan en los siguientes apartados.

\subsection{Criterios de implementación}

Para poder analizar de manera pura la diferencia entre los dos paradigmas, se procuró que las implementaciones en lenguaje C fueran lo mas intuitivas posibles; es decir, una traducción fiel a la descripción en lenguaje natural del comportamiento del filtro. No se utilizaron optimizaciones algorítmicas sofisticadas. Esto acarrea el beneficio adicional de simplificar la interpretación del código objeto generado por los compiladores, y les provee mayor libertad para realizar paralelización u otras optimizaciones sin necesariamente respetar el esquema planteado.

Por otro lado, todas las implementaciones en ASM se desarrollaron intentando minimizar los accesos a memoria, para evitar el cuello de botella de la arquitectura de Von Neumann. Una vez dentro del ciclo, sólo se accede a memoria para la lectura de la imagen fuente o la escritura de la imagen destino.

% (esto es medio un spoiler)
% Sin embargo en cada implementación se va a analizar si esto fue una decisión acertada o no.

\subsection{Compiladores utilizados}
			
Todos los programas en lenguaje C se compilaron con los siguientes compiladores:

\begin{itemize}
	\item GNU C Compiler (GCC): Se eligió por ser un compilador libre, conocido, popular, sumamente versatil y porque es capaz de realizar una gran gama de optimizacizaciones, probablemente todas aquellas que se pueden realizar sin acceder al micro código de Intel.

	\item Intel C++ Compiler (ICC): Este compilador realiza optimizaciones muy avanzadas aprovechando características detalladas de la microarquitectura Intel. Adicionalmente, introduce instrucciones de la familia de extensiones SSE de forma predeterminada; es decir, por defecto es capaz de generar código objeto manifestando paralelismo a nivel de datos.
\end{itemize} 
				
% esto no lo entiendo
Además si bien siempre se compiló indicándole al compilador que use instrucciones SSE4.3 además se hicieron 2 versiones distintas con cada compilador: Una con optimizaciones agresivas y otra sin ellas. Para la primera usó el flag -Ofast, mientras que para la segunda se dejó el comportamiento predeterminado.


\subsection{Entorno de experimentación y método de medición}

	A la hora de realizar las mediciones de tiempo se intentó armar el ambiente mas
ameno posible. Para esto antes de realizar los tests se mataron todas las aplicaciones
no vitales para el sistema operativo, incluída la interfaz gráfica, acceso a internet y
toda esa clase de cosas (init 1, kill -9 -1). Además se desconectaro todos los periféricos
innecesarios.

	Las mediciones se repitieron entre 50 y 100 veces (según el filtro). Estos números se
decidieron en base a prueba y error. Eliminando los valor atípicos de esa cantidad
de valores hacia arriba la variación de la media era despreciable (menor al 0.5)

	Para eliminar los valores atípicos se utilizó el procedimiento de rango intercuartil. Es decir
que se eliminaron aquellos elementos que estuvieran en los dos cuartiles externos.

% le voy a enchufar otra fotito aca
% \begin{figure}[h]
% \begin{center}
%   \includegraphics[scale=0.5]{secciones/consideraciones/imagenes/cuartiles.png}
% \end{center}
% \caption{Representación gráfica de la \footnote{Fuente: http://commons.wikimedia.org/wiki/File:Lqr\_with\_quantile.png}}
% \label{fig:cuariles}
% \end{figure}

	Con el resto de los valores se realizó un promedio. Es muy importante notar que una vez eliminados los valores
atípicos la variación resulta realmente baja. En general de menos del 2\%. De esta manera se logra poder trabajar
con la media, que es una referencia muy fuerte, y además estar muy cerca del mínimo valor obtenido.

	

	


	\newpage

	\section{Filtro de color}
		\subsection{Descripción del filtro}

El filtro de color es una transformación sobre imágenes a color que tiene el efecto de decolorizar o pasar a escala de grises todos los píxeles de la entrada cuyo color no esté dentro de un rango de colores especificado. En la figura \ref{fig:filtro-color-ejemplo} se observa un ejemplo de funcionamiento típico.

\begin{figure}[h]
\begin{center}
  \includegraphics[scale=0.4]{secciones/filtro_color/imagenes/matecocido.jpg}
  \includegraphics[scale=0.4]{secciones/filtro_color/imagenes/matecocido-fcolor.jpg}
\end{center}
\caption{Imagen antes y después de aplicar el filtro de color con color principal rojo.}
\label{fig:filtro-color-ejemplo}
\end{figure}

La forma en la cual se especifica el rango de colores que deberá permanecer inmutado es mediante la elección de un color principal, cuya codificación RGB se denota con (\param{rc}, \param{gc}, \param{bc}), y un parámetro umbral \param{threshold}. Una vez determinados estos valores, un píxel de la imagen fuente será actualizado por el filtro únicamente si cumple:


\begin{equation} \label{filtro-color-condicion}
\vectornorm{(r, g, b) - (rc, gc, bc)} > threshold 
\end{equation}

donde (\param{r}, \param{g}, \param{b}) es la codificación en RGB del píxel. En particular, de cumplirse esta condición, los tres canales se actualizan de la siguiente forma:

$$ r' = b' = g' = \frac{r + g + b}{3} $$

De esta última expresión se desprende que el color de los píxeles alterados pasa a estar en la escala de grises, ya que los tres canales toman igual valor. Como una observación adicional, queda claro mediante esta especificación que el filtro actúa de forma localizada en sobre cada píxel; su suceptibilidad a ser modificado y su nuevo valor dependen únicamente de su propio valor, y no del de sus vecinos.

\subsection{Implementación en lenguaje C y lenguaje ensamblador}

La implementación en C del filtro se realizó de la forma más sencilla e intuitiva posible; mediante un ciclo que visita una vez a cada píxel de la imagen, de izquierda a derecha y de arriba a abajo, evaluando la condición y modificando su valor de ser necesario. Como única optimización elemental, se modificó la condición (\ref{filtro-color-condicion}) por la siguiente condición equivalente:

$$\vectornorm{(r, g, b) - (rc, gc, bc)}^2 = (r - rc)^2 + (g - gc)^2 + (b - bc)^2 > threshold^2$$

La modificación evita el cálculo de una raíz cuadrada, sin incurrir en el riesgo de exceder el rango del tipo de datos utilizado ya que el máximo \param{threshold} que no hace a la condición trivialmente falsa es $\sqrt{195075} \approx 441$ (el valor máximo que puede tomar la expresión de la izquierda es $255^2 + 255^2 + 255^2 = 195075$), por lo que cualquier valor mayor se puede reducir a $442$.

La implementación en lenguaje ensamblador mantiene esencialmente el mismo procedimiento, con la salvedad de que fue adaptado para procesar cuatro píxeles en simultáneo mediante el uso de operaciones SIMD. Se recorre la matriz en el mismo órden descripto previamente, realizando lecturas de 16 bytes por iteración (equivalente a 5 píxeles y un byte sobrante).

La necesidad de limitar el procesamiento simultáneo a cuatro píxeles se desprende del hecho de que, como se dijo antes, la expresión que mide la distancia entre un píxel y el color principal puede tomar valores en el rango $0, ... \,195075$. Este último valor no cabe en un entero de 16 bits, precisándose un \tipo{double word} para almacenar ese resultado temporal; esto implica un total de hasta cuatro valores de distancia en un registro XMM. En cuestiones de espacio, hubiese sido equivalente utilizar \tipo{floats} ya que también caben hasta cuatro por registro de 128 bits.

De esta forma, el procedimiento realizado en cada iteración consiste en (para cada uno de los cuatro píxeles en simultáneo) comparar el valor de la distancia con el \param{threshold}, y calcular el promedio de los tres canales, actualizando luego sus valores según la siguiente expresión informal:

$$\text{valor\_original} \land \lnot \text{cumple\_condicion} + \text{promedio} \land \text{cumple\_condicion}$$

Esto permite expresar el equivalente a una expresión del tipo \textbf{if}-\textbf{then}-\textbf{else}, en el lenguaje del procesamiento simultáneo. El cómputo de los flags con el resultado de las comparaciones y del promedio se puede describir mediante el pseudo-código de la figura \ref{fig:pseudocodigo-filtro-color}.

\begin{figure}[h]
	\begin{mdframed}
	\begin{center}
		\begin{lstlisting}
		prom := [0,0,0,0] 			// enteros doubleword
		dist := [0,0,0,0]
	
		desempaquetar el rojo de cada pixel a un double word
		rojos := [r4, r3, r2, r1]
	
		prom += rojos
		rojos -= [rc, rc, rc, rc]
		rojos *= rojos
		dist += rojos
	
		repetir para verdes
		repetir para azules
	
		prom := prom / 3
		dist := dist > [threshold, threshold, threshold, threshold]
		empaquetar prom y dist a formato pixeles

		resultado := datos AND NOT dist
		resultado += prom AND dist
		\end{lstlisting}
	\end{center}
	\end{mdframed}
	\caption{Pseudo-código de la implementación en lenguaje ensablador del filtro color.}
	\label{fig:pseudocodigo-filtro-color}
\end{figure}

Para la correcta implementación de este procedimiento es importante contemplar el caso excepcional que sucede en la última iteración, cuando falta procesar los últimos cuatro píxeles de la imagen. De realizarse una lectura desde el comienzo del píxel, se leería junto a los píxeles restantes un total de 4 bytes de memoria posteriores al fin de la imagen. Por esta razon, la última iteración se dejó fuera del ciclo principal, realizando un retroceso de 4 bytes en el puntero de lectura, y ejecutando una versión levemente modificada del cuerpo de ciclo de forma tal que procese los píxeles en los últimos 12 bytes del registro, en vez de los primeros.

\subsection{Optimizaciones en código C}
\label{sub:filtro-color-optimizaciones-c}

El código de la implementación en lenguaje C se compiló con los distintos flags de optimización descriptos en la sección \ref{}. Los resultados en cuestión de performance se pueden ver en la figura \ref{}, la cual muestra una comparación entre la cantidad de clocks consumidos durante la ejecución de cada versión.

TODO:: Discutir un poco la figura.

Adicionalmente, se analizaron las diferencias entre el código generado con compilación estándar y el código generado con optimización de tipo O1. La diferencia entre ambos evidencia inmediatamente una característica subóptima del código estándar; \emph{todas} las variables locales se alojan en la pila, incluso habiendo disponibilidad de registros. Los registros se utilizan únicamente como variables temporales, y al final de cada extracto de código se guarda el resultado en un espacio de la pila. El código generado con O1, en cambio, ahorra gran cantidad de esos accesos a memoria. Este comportamiento se ejemplifica con el siguiente extracto (figura \ref{fig:codigo-objeto-filtro-color}).

% esto está atado con alambre...
\begin{figure}[h]
	\begin{mdframed}
	\begin{center}
		\begin{lstlisting}
		mov  eax, DWORD PTR[rbp-0x10]	| imul   r15d,r15d
		mov  edx,eax			| imul   r14d,r14d
		imul edx, DWORD PTR[rbp-0x10]	| add    r14d,r15d
		mov  eax, DWORD PTR[rbp-0xc]	| imul   ebx,ebx
		imul eax, DWORD PTR[rbp-0xc]	| add    ebx,r14d
		add  edx,eax			|			
		mov  eax, DWORD PTR[rbp-0x8]	|			
		imul eax, DWORD PTR[rbp-0x8]	|			
		add  eax,edx			|			
		mov  DWORD PTR [rbp-0x4],eax	|			
		\end{lstlisting}
	\end{center}
	\end{mdframed}
	\caption{Código objeto generado con compilación por default (izquierda), y con optimización de tipo O1 (derecha).}
	\label{fig:codigo-objeto-filtro-color}
\end{figure}

Sin embargo, la lógica de flujo del código generado es equivalente. Es decir, las operaciones realizadas y el orden en que se realizan son iguales; se mantiene la forma de recorrer la imagen y el procesamiento píxel por píxel.

\subsection{Optimización en código ASM - Desenrollado de ciclos}
\label{sub:filtro-color-optimizaciones-asm}

Sobre la implementación en lenguaje ensamblador ya descripta, se realizaron dos sucesivas modificaciones para estudiar la técnica conocida como desenrollado de ciclos.

% Describir ocmo funciona el filtro.	
% Como se implementó en c
% Cómo se implementó en asm
	% - explicar el algorítmo
	% - explicar el caso borde
	% - explicar las optimizaciones realizadas. Loop unrolling.
	% - particularidades del filtro. Div 3. Se hicieron todas las cuentas con ints pero al final hubo que pasar a float para poder dividir por 3.
%--Gráfico--  (Performance)
% C vs ASM
% C vs O1, O2, O3
% ASM vs loop unrolling x2 vs loop unrolling 4
% C con y sin condicionales


% Comparar estructura en c y en asm.
	% - En c es un loop que procesa un pixel por vez con un condicional adentro (lo que hace tarde distinta cantidad de tiempo para distintas imágenes).
	% - En asm procesa de a 4 pixeles por vez y hace el mismo proceso independientemente de la imagen.
	% - ¿Que pesa más?¿Procesamiento o acceso a memoria?
	% - Diferencias estructurales
		% - Comparación líneas de código



	\newpage

	\section{Filtro Miniature}
		\subsection{Descripción del filtro}
\label{sub:miniature_descripcion}

El filtro miniature es un caso particular de la familia de filtros por convolución, en los cuales se procesa una imagen realizando una convolución entre esta y una matriz determinada denominada \emph{kernel}. Las características del \emph{kernel} determinan el efecto resultante sobre la matriz; en este caso, el efecto logrado se conoce como desefoque gaussiano (se puede ver un ejemplo en la figura \ref{}), y se obtiene mediante el siguiente \emph{kernel}:

\[ \frac{1}{600} *  \left( \begin{array}{ccccc}
  1 &   5 &  18 &   5 &   1 \\
  5 &  32 &  64 &  32 &   5 \\
 18 &  64 & 100 &  64 &  18 \\
  5 &  32 &  64 &  32 &   5 \\
  1 &   5 &  18 &   5 &   1 \end{array} \right)\]

Operativamente, la imagen es procesada mediante una actualización píxel a píxel, reemplazando el valor de cada canal de color del píxel por una combinación lineal del valor de sus vecinos, cada uno ponderado por un coeficiente determinado por un elemento de la matriz. Se toma el elemento central del \emph{kernel} como coeficiente del píxel que se está procesando, y la posición relativa a los demás elementos de la matriz determina a cuál vecino corresponde en la combinación lineal. El factor $\frac{1}{600}$ es una constante de normalización que garantiza que el resultado sea un valor en el rango $[0, 255]$.

\begin{figure}[h]
\begin{center}
  \includegraphics[scale=0.2]{secciones/filtro_miniature/imagenes/neo.jpg}
  \includegraphics[scale=0.2]{secciones/filtro_miniature/imagenes/neo-miniature.jpg}
\end{center}
\caption{Imagen antes y después de aplicar el filtro miniature con parámetros de banda $0.08$ y $0.25$, y un total de 20 iteraciones.}
\label{fig:filtro-miniature-ejemplo}
\end{figure}

Se incorpora una leve modificación al modelo típico de filtro por convolución, limitando el efecto del filtrado a dos bandas dentro de la imagen; una banda superior, y una banda inferior, dejando la banda central de la imagen inalterada. Se especifican dos parámetros $0 < $ \param{topPlane} $<$ \param{bottomPlane} $ < 1$, de forma tal que las bandas quedan determinadas de esta forma:

\begin{center}
	\raggedright
	\hspace{100pt}\textbf{banda superior:} 	\hspace{10pt}filas 0 ... $topPlane * altura$\\
	\hspace{100pt}\textbf{banda media:} 		\hspace{22pt}filas $topPlane * altura + 1$ ... $bottomPlane * altura - 1$\\
	\hspace{100pt}\textbf{banda baja:} 		\hspace{31pt}filas $bottomPlane * altura$ ... $altura - 1$\\
\end{center}

Adicionalmente, se realizan múltiples iteraciones de filtrado, reduciendo el ancho de cada banda luego de cada iteración por:

\begin{center}
	\raggedright
	\hspace{100pt}$\Delta$\textbf{banda superior:} 	\hspace{10pt}$topPlane * altura / cantIteraciones$\\
	\hspace{100pt}$\Delta$\textbf{banda baja:} 		\hspace{31pt}$(1 - bottomPlane) * altura / cantIteraciones$\\
\end{center}

En el caso de los píxeles del borde, donde no existe el vecindario completo, se optó por exceptuarlos del filtrado, simplificando el procedimiento al saltear las dos primeras y dos últimas filas y columnas.

\subsection{Implementación en lenguaje C y lenguaje ensamblador}
\label{sub:miniature_implementaci_n_en_c}

El filtro se implementó en C mediante un ciclo que visita una vez por iteración a cada píxel de la banda superior y de la banda inferior, y actualizando el valor de sus tres canales de color por la combinación lineal descripta previamente.

Para este filtro, la implementación intuitiva en C resultó particularmente sugerente a la necesidad de buscar formas de optimizar el procedimiento, dado que cada píxel se lee de la memoria hasta 25 veces (una vez por cada vecino) resultando en un tiempo de ejecución prolongado; y además, el tipo de procesamiento realizado por cada píxel es el cómputo de una combinación lineal, para lo cual los procesadores están altamente optimizados, desperdiciando los beneficios del hardware.




\begin{comment}
	Se realizó todo con ints en lugar de floats justificar
		- En asm ocupan menos lugar en los xmm (O sea que se pueden procesar de a más a la vez).
	Describir el algoritmo en c.
		- Se hardcodea la matriz
		- Se hardcodea la suma
		- Se procesa de a un pixel a la vez (los tres canales en cada iteraicón)
	Describir el algoritmo en asm (con dibujito).
		- Especificar diferencias.
		- Que la memoria se levanta en 2 etapas y se reutiliza.
		- Se procesan 4 pixeles por iteración pero de a 2 a la vez.
			- Mencionar el tema de los rangos. Para explicar por qué es necsario esto y además por qué conviene trabajar con enteros de word y no con float.
		- Caso borde, ¿Por qué no existe?.
		- Tiempos intermedios (cálculo y acceso a memoria)
			- Como se realizó esta medición y por qué (comentando código).
			- Explicar lo que representa realmente el gráfico de torta y que tiene error.
\end{comment}

	\newpage

	\section{Decodificación esteganográfica}
		\begin{comment}
	- Comparación c y asm.
		- La mayor diferencia es que se levanta de a más.
	- El tiempo en asm es súper constante porque sólo depende de size. En c mas o menos, explicar por qué.
	- Hay un caso límite. Explicar. Aunque se trata como un chorizo.
	- Explicar optimización del acceso a memoria.
	- Explicar por qué no se puede hacer tanta ejcución fuera de orden. Hay dependencia de datos.
	- ¿Que pesa más?¿Procesamiento o acceso a memoria?

\end{comment}

\subsection{Descripción del filtro}

	El filtro consite en extraer un mensaje codificado detro de la imagen conociendo
, por supuesto la menera en que este está codificado.

	La codificación consiste en cambiar cambiar en cada byte de la imagen los
dos bits menos significativos por los bits del mensaje. Es decir que cada
byte del mensaje de codifica en 4 bytes de la imagen (2 bits en cada imagen).

	Es decir que la imagen está alterada, pero está los cambios son tan leves
que no son visibles al ojo humano. Es mucho mayor la variación de colores que
se produce por ver la imagen en diferentes monitores que la que se produce
por cambiar estos bits.

	Los pares de bits, además tienen una breve codificación individual. Que
se determina mirando los bits 2 y 3 de cada Byte. Es decir que uno sabe
como interpretar los dos bits menos significativos mirando los siguientes
2 bits.


\subsection{Implementaciones}


	Para decode se usaron 3 implementaciones: Una escrita en lenguaje C.
Una escrita en lenguaje ensamblador y otra escrita en lenguaje ensamblador
intentando usar la máximo los beneficios de la ténica de software pipelining.

	El algorítmo de C al igual que en los otros filtros es lo mas intuitivo posible.
Basicamente se lee de a un byte de la imagen. Mediante máscaras se filtran los dos bits
menos significativos y los siguientes 2 bits menos significativos.

	Los bits 2 y 3 se comparan en un switch para saber como procesar
a los bits 0 y 1. Una vez que se procesaron los bits 0 y 1 se los guarda se los
mueve a izquierda la cantida de lugares adecuada (0, 2, 4 o 6) según sea el primer
par de bits de ese bytes, el segundo, el tercero o el cuarto. Luego se van acumulando
esos resultados parciales y cuando se tiene un byte entero se lo guarda en el vector
destino.

	Cabe aclarar que a la hora de implementar en ensamblador este filtro provee bastantes
facilidades. Todos los corrimientos que hay que hacer son múltiplos de dos y la cantidad
de datos que hacen falta decodidificar para formar un byte es potencia de 2. Estas
cosas facilitaron mucho el trabajo. Además en ningún momento se necesita saber la
posición de los pixeles ni nada por el estilo, por lo que la matriz fuente se puede
tomar sencillamente como un gran vector.

	El proceso, entonces, es sencillo:

\begin{enumerate}

	\item Se traen datos de la matriz fuente a un registro xmm. Una vez que
	están en registro estos datos se copian a otro xmm

	\item En una de las copias se conservan los bits 0 y 1. En otra los bits 0 y 3.

	\item A todos los bits se los hace pasar por el proceso de sumar, restar
	y negar. Sin embargo lo que se hace es usar una máscara para poner diferentes operandos
	en cada uno. Por ejemplo cuando se realiza la suma sólo aquellos valores
	a los que corresponde sumar tienen como segundo operando un uno, el resto tiene
	un cero. Para que esto quede así se repiten los siguientes pasos para cada
	posicle operación:

	\begin{enumerate}

		\item Se compara el registro que tiene los bits 2 y 3 con una
		máscara que contiene la referencia a la operación a realizar repetida
		en cada byte. Por ejemplo la máscara de negar tiene 0x0C en todos sus bytes.

		\item Cada vez que se compara con una máscara luego se realiza una conjunción
		con otra máscara que tiene el segundo operando de la operación a relizar.
		Por ejemplo si lo que hay que hacer es la dato sumarle uno entonces
		Se usa una máscara de unos. En el resto de las posiciones queda cero.

		\item Por último se realiza la operación entre los datos (es decir los bits
		menos significativos de los bytes traídos de la imagen) y la máscara recién
		creada.

	\end{enumerate}

	\item Una vez realizado 4 cuartetos de pares de bits alineados. Para que ocupen las
	posiciones adecuadas, entonces, lo qeu se hace es realizar 4 copias, shiftear de
	manera empaquetada , 0 , 2 , 4 y 6 lugares respectivamente y por último sumar todo.
	
	\item Ahora se tienen 4 bytes armados pero desordenados. Es necesario tenerlos todos
	en la parte baja del registro para poder grabar a memoria. Esto se hace mediante un shuffle
	de bytes.

	\item Al final del ciclo existe un caso complicado. El tamaño del string de salida
	no tiene por qué ser múltiplo de 4 (hasta ahora se estuvieron insertando de a 4
	bytes a la vez). Para resolver esto se implementó una lógica especial
	intentando que reste la menor cantidad de performance posible. Sólo se entra al código
	que describe esta lógica en la anteúltima y en la última iteración.

\end{enumerate}


\subsection{Optimizaciones}

	Se intentó utilizar el entubado de código ($software pipelining$) de diversas maneras.
En un principio se dividió el proceso en 4 etapas: Acceso a memoria por un lado
y por otro los 3 entradas del switch. Sin embargo esto no produjo mejoras en la performance,
por el contrario disminuyó. Por lo tanto se bajó un poco el nivel de ambición y se realizaron
sólo 2 etapas. Por un lado el acceso a memoria y por el otro lado el proceso de los datos. La idea
era lograr que el proceso siga mientras se realiza el acceso a memoria en lugar de tener que esperar.
El proceso fue el siguiente:

	Antes de entrar al ciclo se traen los datos necesarios para hacer la priemra iteración.
Una vez adentro del ciclo se traen los datos necesarios para la segunda iteración, pero
se trabaja con los datos que ya se trajeron antes. Luego al finalizar el ciclo.
Se mueven los datos traídos al principio del ciclo al registro en el cuál se trabaja.

	El resultado de esto es que las operaciones no tienen dependencias a nivel de datos con
la lectura de memoria, por lo tanto el procesador no tienen que esperar a que se termine el ciclo
de lectura para poder empezar a trabajar con los datos.

	A nivel de código la optimización es súper sencilla. Mucho mucho más que la enterior.
Bastó con agregar unas pocas líneas de código en el lugar adecuado para aumentar en un 15\% la performance
con respecto al código original. Un par de pruebas nos mostraron que el lugar óptimo para hacer el último
movimiento es lo mas cerca del final del ciclo posible.

\subsection{Rendimiento}


---Insertar gráfico---

	

	\newpage

	\section{Conclusiones}
		\begin{comment}
	- ¿Valió la pena?
		- Evaluar costo de implementación vs performance.
		- Evaluar si el aumento de performance tiene o no sentido. (Caso decode, es bien al pedo. Sin embargo en aplicaciones de tiempo real o de uso masivo tiene mucho sentido porque el aumento es muy significativo).
		- 
	- ¿Cuánto pesa el uso óptimo del hardware vs el algorítmo?
		- Explicar por qué el algoritmo de fondo es el mismo concluyendo que un mejor uso del hardware puede lograr incrementos sumamente significativos.
	- ¿Se llegó a cambiar el orden de magnitud?
		- Si bien obviamente no se llegó orden de complejidad si aumenta el orden de magnitud en cuanto a velocidad de ejecución. Decode y miniature.

\end{comment}

	A lo largo de este trabajo se analizaron diferentes implementaciones
los filtros bajando al nivel mas bajo posible para explicar los resultados.
Los resultados son bastante visibles, se logró acelerar el tiempo de cómputo
de manera extremadamente significativa, sencillamente haciendo un correcto
aprovechamiento de los recursos brindados por la arquitectura de procesador
usada.

	Antes de sacar conclusiones generales cabe hacer unos breces comentarios
particulares de cada filtro.

\begin{itemize}
	\item \textbf{Fcolor:} En este filtro se obtuvo un resultado bueno.
La implementación en ensablador va mas de 2 veces mas rápido que la implementación
en C. Sin embargo la implementación en ensablandor fue realmente mucho mas costosa
que la implementación en C. Es importante marcar que en el primer
intento este filtro implementado en ensamblador no fue mas rápido que C sino
que hubo que trabarlo bastante para lograrlo. Como conclusión entonces
podríamos decir que la relación costo-beneficio no es tan ventajosa. Vale
la pena realizar un trabajo así sólo si se está trabajando con sistemas
muy críticos.
	\item \textbf{Miniature:} Este filtro es el claro ejemplo de cuando definitivamente
vale la pena ensuciarse las manos con el ensamblador para exprimir los recursos
del procesador. Con las velocidades alcanzadas con la implementación en ensablandor
este filtro incluso se podría utilizar en tiempo real (al menos para ciertos parámetros)
mientras que con los tiempos obtenidos con la implementación en C eso no es mas que una
fantasía lejana.
	\item \textbf{Decode:} Con este filtro pasa algo particular. Los resultados
obtenidos fueron excelentes. La implementación en ensablador es exsesivamente mas rápida
que la implementación en C... Y sin embargo es probable que no valga la pena. Si
se codificara en una gigantezca imagen todo el libro "El ingenioso hidalgo Don Quijote
de la Mancha" este filtro en su implementación en C lo decodificaría en no más de 15
segundos. Definitivamente leer esa historia toma bastante más de 15 segundos, por lo
que definitivamente eso sirve para tiempo real. Por otra parte si se usa en un sistema muy
concurrido este debería ser un sistema con una concurrencia realmente alta para ocasionar
problemas. De esta manera la implementación en ensamblador es excesivamente rápida, pero
este filtro en particular tal vez no sería la elección natural para implementar en
ensamblador a la hora de acelerar un sistema. Sería muy poco probable
que se lo identifique como un punto crítico.
	Dejando de lado eso es interesante marcar que realmente el aumento de rendimiento
en este filtro es muy importante. Eso en parte es porque este filtro es muy ``SIMD friendly''
como se explicó anteriormente. Por lo tanto es interesante tener en cuenta procesos
con la estructura que tiene este filtro son fáciles de implementar en ensamblador
y obtener excelentes resultados.
\end{itemize}

	Aclaradas estas cuestiones el siguiente análisis que cabe hacer
es el de ''Complejidad algorítmica vs Optimización de implementación''.
Todos los filtros tuvieron cambios algorítmos a la hora de adaptarlos
al procesamiento SIMD, sin embargo esos cambios no fueron estructurales.
Sencillamente fueron los cambios indispensables para esta clase de procesamiento.
La única excepción, tal vez, fue Miniature. Sin embargo el cambio
principal que se realizó fue sobre como recibir los datos. El procesamiento
de sigue siendo muy similar. De todas maneras la complejidad del algorítmo
sigue siendo la misma, para todos los filtros. Sin embargo mediante
optimizaciones de bajo nivel se logró incluso cambiar el orden de magnitud
en el tiempo (Caso fcolor y decode).

	La contracara de esto es como realmente aumenta la dificultad en la
implementación. Esto hace que no sea viable implementar grandes sistemas
en ensablandor utilizando todas las optimizaciones de bajo nivel posibles.
Sin embargo si es viable identificar cuellos de botella o puntos críticos
en un sistema y aplicar ahí toda la optimización posible. Esa práctica
puede aumentar de manera sumamente significativa el rendimiento del sistema.


	
\end{document}
