% ** Cuidado: El indice se actualiza al compilar 2 veces. **

% -- Configuración --
\documentclass[10pt, a4paper]{article}
\usepackage[paper=a4paper, left=1.5cm, right=1.5cm, bottom=1.5cm, top=1.5cm]{geometry}
\usepackage[utf8]{inputenc}
\usepackage[spanish]{babel}
\usepackage{float}
% \usepackage[colorlinks=true, linkcolor=blue]{hyperref}
\usepackage{graphicx}
\usepackage{titling}
\usepackage{tikz}
\usepackage{dsfont}
\usepackage{amsmath}
\usetikzlibrary{chains,fit,shapes}
\newcommand{\ig}[1]{\includegraphics[width=\textwidth]{#1}}
\newcommand{\RNum}[1]{\uppercase\expandafter{\romannumeral #1\relax}}
\usepackage{parskip}
 
% -- Documento --
\begin{document}

\setlength{\parindent}{25pt}

	% -- Carátula --
	
	\thispagestyle{empty}
	\title{%
	\huge{Organización del Computador \RNum{2}}\\
	\vspace{4mm}
	\large{Trabajo Práctico 2}
	}
	\date{\vspace{5mm}$2^{\mathrm{do}}$ cuatrimestre de 2013}

	\author{
		\\
		{\rm Emilio Almansi }\\
		\small{ealmansi@gmail.com}
		\and
		\\
		{\rm Federico Suárez}\\
		\small{elgeniofederico@gmail.com}
		\and
		\\
		{\rm Miguel Duarte}\\
		\small{miguelfeliped@gmail.com}
	} % end author
	\maketitle

	
	\vspace{10mm}
	% -- Resumen --
	\begin{abstract}
		\centering{%
			En este documento se analiza las diferencias en rendimiento encontradas para las implementaciones de algunos 
			filtros de imágenes al utilizar instrucciones SIMD de la arquitectura Intel 64.
		}
	\end{abstract}
	\vspace{10mm}

	% -- Indice --

	\tableofcontents

	\newpage
	% -- Contenido --
	% Se edita cada seccion por separado, no hay que cambiar nada acá.
	\section{Introducción}
		% Explicar en que consitió el trabajo (brevemente).
	% - Explicar levemente los criterios usados para la paralelización
	% - Por qué es buena idea usar este tipo de procesamiento en filtros de imágenes. (todos los pixeles reciben exactamente el mismo procesamiento y así en una sola ráfaga se peude levantar y procesar muchos).
	% - Explicar la hipótesis (debería ser n veces mas rápido porque zarasa).
	% - Los tres filtros se programaron pensando en minizar los accesos a memoria incluso en situaciones donde producía cálculo extra.
	% - Se intentó estudiar si vale la pena o no el esfuerzo adicional de desarrollo y debuggeo.

 


	
	\section{Filtros Implementados}
		% comparamos implemtaciones del mismo filtro (versión mas intuitiva posible de c vs asm con SIMD). Esto permite analizar mejor la optimizaciones del compilador. Sobre todo las de icc que tiene introducción automática de SIMD
% En todos los casos se realizó el gráfico C vs ASM (cantidad de clocks y cantidad de líneas de código).
% Como medimos. (menor vs promedio)
% Como se compiló. Compilamos con gcc, con icc, y con diferentes flags.
	% - (aclarar cuales y por qué)
	% - Explicar por qué se puso ICC y no otro(introduce SIMD automáticamente y se supone que es el compilador mas optimo para la arquitectura AMD-64 con procesadores intel).
% cómo se ejcutó. (Mínimas interrupciones, init 2 y kill -9 -1).
% Análisis del código. (object dump).

	\newpage
	\subsection{Filtro de color}
		% Describir ocmo funciona el filtro.	
% Como se implementó en c
% Cómo se implementó en asm
	% - explicar el algorítmo
	% - explicar el caso borde
	% - explicar las optimizaciones realizadas. Loop unrolling.
	% - particularidades del filtro. Div 3. Se hicieron todas las cuentas con ints pero al final hubo que pasar a float para poder dividir por 3.
%--Gráfico--  (Performance)
% C vs ASM
% C vs O1, O2, O3
% ASM vs loop unrolling x2 vs loop unrolling 4
% C con y sin condicionales


% Comparar estructura en c y en asm.
	% - En c es un loop que procesa un pixel por vez con un condicional adentro (lo que hace tarde distinta cantidad de tiempo para distintas imágenes).
	% - En asm procesa de a 4 pixeles por vez y hace el mismo proceso independientemente de la imagen.
	% - ¿Que pesa más?¿Procesamiento o acceso a memoria?
	% - Diferencias estructurales
		% - Comparación líneas de código



	\newpage
	\subsection{Miniature}
		\begin{comment}
	Se realizó todo con ints en lugar de floats justificar
		- En asm ocupan menos lugar en los xmm (O sea que se pueden procesar de a más a la vez).
	Describir el algoritmo en c.
		- Se hardcodea la matriz
		- Se hardcodea la suma
		- Se procesa de a un pixel a la vez (los tres canales en cada iteraicón)
	Describir el algoritmo en asm (con dibujito).
		- Especificar diferencias.
		- Que la memoria se levanta en 2 etapas y se reutiliza.
		- Se procesan 4 pixeles por iteración pero de a 2 a la vez.
			- Mencionar el tema de los rangos. Para explicar por qué es necsario esto y además por qué conviene trabajar con enteros de word y no con float.
		- Caso borde, ¿Por qué no existe?.
		- Tiempos intermedios (cálculo y acceso a memoria)
			- Como se realizó esta medición y por qué (comentando código).
			- Explicar lo que representa realmente el gráfico de torta y que tiene error.
\end{comment}


	\newpage
	\subsection{Decodificación esteganográfica}
		\begin{comment}
	- Comparación c y asm.
		- La mayor diferencia es que se levanta de a más.
	- El tiempo en asm es súper constante porque sólo depende de size. En c mas o menos, explicar por qué.
	- Hay un caso límite. Explicar. Aunque se trata como un chorizo.
	- Explicar optimización del acceso a memoria.
	- Explicar por qué no se puede hacer tanta ejcución fuera de orden. Hay dependencia de datos.
	- ¿Que pesa más?¿Procesamiento o acceso a memoria?

\end{comment}

	\newpage
	\section{Conclusiones}
		\begin{comment}
	- ¿Valió la pena?
		- Evaluar costo de implementación vs performance.
		- Evaluar si el aumento de performance tiene o no sentido. (Caso decode, es bien al pedo. Sin embargo en aplicaciones de tiempo real o de uso masivo tiene mucho sentido porque el aumento es muy significativo).
		- 
	- ¿Cuánto pesa el uso óptimo del hardware vs el algorítmo?
		- Explicar por qué el algoritmo de fondo es el mismo concluyendo que un mejor uso del hardware puede lograr incrementos sumamente significativos.
	- ¿Se llegó a cambiar el orden de magnitud?
		- Si bien obviamente no se llegó orden de complejidad si aumenta el orden de magnitud en cuanto a velocidad de ejecución. Decode y miniature.

\end{comment}


	
\end{document}
